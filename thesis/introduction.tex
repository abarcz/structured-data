
\chapter{Introduction}
The Graph Neural Network model is a connectionist classifier capable of classifying graphs. Most of the other existing neural network-based graph classifiers, such as RAAM~\cite{pollack1990recursive} or  LRAAM~\cite{sperduti1994labelling} and all solutions basing on them are capable of processing certain types of graphs only, in most cases DAGs (directed acyclic graphs) or DPAGs (directed positional acyclic graphs).
Several solutions were invented to deal with cyclic graphs, such as introducing a delay in the LRAAM encoding tree~\cite{goulon2005hopfield} or techniques mapping cyclic directed graphs to "recursive equivalent" trees~\cite{bianchini2003backpropagation}.
The problem of nonpositional graphs was also addressed by several authors, either by creating domain-specific encodings used to enforce a defined order on graph nodes~\cite{ivanciuc2003canonical} or by introducing various modifications to the classifier~\cite{bianchini2005recursive}.
However, most of the solutions dealing with cyclic and nonpositional graphs either complicate the classifier model, enlarge the input dataset or (in case of cycles) may result in information loss. The Graph Neural Network model can directly process most types of graphs, including cyclic, acyclic, directed, undirected, positional and nonpositional, which makes it a flexible solution.
There is a conceptual similarity between the GNN model and recursive neural networks~\cite{frasconi1998general}, however, the GNN model adapts a novel learning and backpropagation schema which simplifies processing different types of graphs.
This thesis describes the steps of implementation of a GNN classifier, including some details which were not described in the original article~\cite{scarselli2009graph}. The classifier was implemented in GNU Octave with two ideas in mind: providing a simple interface (similar to that of the Neural Networks toolbox) and maximum flexibility, that is the possibility of processing each kind of data that the theoretical model could deal with. The process of training a GNN and classification results were presented in detail.

