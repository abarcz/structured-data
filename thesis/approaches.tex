
\chapter{Approaches}
\noindent A model considered as fully capable of processing structured data, should be able to:
\begin{enumerate}
	\item build data representation
	\begin{enumerate}
		\item minimal
		\item exploiting sufficiently the structure of data
		\item adequate for subsequent processing (classification, regression)
	\end{enumerate}
	\item perform classification / regression on the structured data
	\begin{enumerate}
		\item taking into consideration the structure encoded in the representation
		\item with a high level of generalisation
	\end{enumerate}
\end{enumerate}
These two main tasks are often intertwined with each other, as a classification procedure may affect the procedure of representation building and vice versa. It is also possible for a model to focus only on representation building, while leaving the task of processing to a common statistical classifier, as support vector machine. Two main approaches to structured data processing are the \emph{symbolic} and the \emph{connectionist} approach. The first one originates in the artificial intelligence domain and focus on inferring relationships by means of inductive logic programming. The \emph{connectionist} approach focus on modelling relationships with the use of interconnected networks of simple units. The different models originating from these two approaches are:
\begin{enumerate}
	\item inductive logic programming
	\item evolutionary algorithms
	\item probabilistic models: Bayes networks and Markov random fields
	\item graph kernels
	\item recursive neural networks
\end{enumerate}
The main focus of this thesis is on the connectionist models based on recursive neural networks, as the models making the fewest assumptions about the domain of the dataset and thus providing a potentially most general method of processing structured data.
