\documentclass[11pt,a4paper]{article}
\usepackage{polski}
\usepackage[utf8]{inputenc}
\usepackage[colorlinks=true,linkcolor=black,urlcolor=blue,citecolor=RoyalBlue]{hyperref}
\usepackage[usenames,dvipsnames]{color}
\usepackage{alltt}
\usepackage{booktabs} % eleganckie tabelki
\usepackage{graphicx}
\usepackage{bm}		% bold math symbols

\newcounter{liczp}
\newenvironment{example}{\refstepcounter{liczp}{\noindent
{\bf Przykład~\theliczp:}\,}}

\usepackage{parcolumns}	% two-column paragraphs
\usepackage{subfigure}
\usepackage{floatrow}
% Table float box with bottom caption, box width adjusted to content
\newfloatcommand{capbtabbox}{table}[][\FBwidth]

\addtolength{\textwidth}{4cm}
\addtolength{\hoffset}{-2cm}
\addtolength{\textheight}{4cm}
\addtolength{\voffset}{-2cm}
\date {\today}
\author {
Aleksy Barcz\\
Instytut Systemów Elektronicznych
}
%\keywords{teoria wrażliwości, sieci neuronowe}

\title{\vspace{60mm} \textbf{Sprawozdanie z pracowni naukowej} \\ 2014L}
\begin{document}
\maketitle
\thispagestyle{empty}

\newpage

\section{Metoda szybkiego uczenia sieci RAAM z wykorzystaniem metody wrażliwościowej}
Wykonane badania pozwoliły przyspieszyć proces uczenia sieci RAAM na danych strukturalnych (analiza form gramatycznych zdań w języku angielskim), przy użyciu metody wrażliwościowej, stosowanej dotąd tylko do uczenia zwykłych sieci neuronowych. Opis metody oraz uzyskane wyniki zostały zaprezentowane w artykule \emph{Fast learning method for RAAM based on sensitivity analysis}, zaakceptowanym do druku w materiałach pokonferencyjnych konferencji \emph{XXXIV-th IEEE-SPIE Joint Symposium Wilga 2014}, \url{http://wilga.ise.pw.edu.pl/node/744}.

\section{Zastosowanie grafowych sieci neuronowych (GNN) do klasyfikacji siatek Bravais}
Wykonane badania pozwoliły na zastosowanie grafowych sieci neuronowych do trójwymiarowych danych opisujących struktury krystaliczne jako siatki Bravais, o wysokiej cykliczności. Dotychczas dane tego rodzaju nie były nigdy analizowane przy użyciu metod uczenia maszynowego jako dane strukturalne, a jedynie jako wektory cech, a badania związane z maszynami grafowymi unikały danych cyklicznych. Opis metody oraz wyniki zostały zaprezentowane na konferencji \emph{8th International Conference on Neural Network and Artificial Intelligence (ICNNAI'2014)} oraz opublikowane w artykule \emph{Graph Neural Networks for 3D Bravais lattices classification}~\cite{barcz2014graph}, \url{http://link.springer.com/chapter/10.1007%2F978-3-319-08201-1_8}.

\bibliography{bib/sensitivity,bib/gnn}
\bibliographystyle{ieeetr}
\end{document}
